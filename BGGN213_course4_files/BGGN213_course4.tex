% Options for packages loaded elsewhere
\PassOptionsToPackage{unicode}{hyperref}
\PassOptionsToPackage{hyphens}{url}
\documentclass[
]{article}
\usepackage{xcolor}
\usepackage[margin=1in]{geometry}
\usepackage{amsmath,amssymb}
\setcounter{secnumdepth}{-\maxdimen} % remove section numbering
\usepackage{iftex}
\ifPDFTeX
  \usepackage[T1]{fontenc}
  \usepackage[utf8]{inputenc}
  \usepackage{textcomp} % provide euro and other symbols
\else % if luatex or xetex
  \usepackage{unicode-math} % this also loads fontspec
  \defaultfontfeatures{Scale=MatchLowercase}
  \defaultfontfeatures[\rmfamily]{Ligatures=TeX,Scale=1}
\fi
\usepackage{lmodern}
\ifPDFTeX\else
  % xetex/luatex font selection
\fi
% Use upquote if available, for straight quotes in verbatim environments
\IfFileExists{upquote.sty}{\usepackage{upquote}}{}
\IfFileExists{microtype.sty}{% use microtype if available
  \usepackage[]{microtype}
  \UseMicrotypeSet[protrusion]{basicmath} % disable protrusion for tt fonts
}{}
\makeatletter
\@ifundefined{KOMAClassName}{% if non-KOMA class
  \IfFileExists{parskip.sty}{%
    \usepackage{parskip}
  }{% else
    \setlength{\parindent}{0pt}
    \setlength{\parskip}{6pt plus 2pt minus 1pt}}
}{% if KOMA class
  \KOMAoptions{parskip=half}}
\makeatother
\usepackage{color}
\usepackage{fancyvrb}
\newcommand{\VerbBar}{|}
\newcommand{\VERB}{\Verb[commandchars=\\\{\}]}
\DefineVerbatimEnvironment{Highlighting}{Verbatim}{commandchars=\\\{\}}
% Add ',fontsize=\small' for more characters per line
\usepackage{framed}
\definecolor{shadecolor}{RGB}{248,248,248}
\newenvironment{Shaded}{\begin{snugshade}}{\end{snugshade}}
\newcommand{\AlertTok}[1]{\textcolor[rgb]{0.94,0.16,0.16}{#1}}
\newcommand{\AnnotationTok}[1]{\textcolor[rgb]{0.56,0.35,0.01}{\textbf{\textit{#1}}}}
\newcommand{\AttributeTok}[1]{\textcolor[rgb]{0.13,0.29,0.53}{#1}}
\newcommand{\BaseNTok}[1]{\textcolor[rgb]{0.00,0.00,0.81}{#1}}
\newcommand{\BuiltInTok}[1]{#1}
\newcommand{\CharTok}[1]{\textcolor[rgb]{0.31,0.60,0.02}{#1}}
\newcommand{\CommentTok}[1]{\textcolor[rgb]{0.56,0.35,0.01}{\textit{#1}}}
\newcommand{\CommentVarTok}[1]{\textcolor[rgb]{0.56,0.35,0.01}{\textbf{\textit{#1}}}}
\newcommand{\ConstantTok}[1]{\textcolor[rgb]{0.56,0.35,0.01}{#1}}
\newcommand{\ControlFlowTok}[1]{\textcolor[rgb]{0.13,0.29,0.53}{\textbf{#1}}}
\newcommand{\DataTypeTok}[1]{\textcolor[rgb]{0.13,0.29,0.53}{#1}}
\newcommand{\DecValTok}[1]{\textcolor[rgb]{0.00,0.00,0.81}{#1}}
\newcommand{\DocumentationTok}[1]{\textcolor[rgb]{0.56,0.35,0.01}{\textbf{\textit{#1}}}}
\newcommand{\ErrorTok}[1]{\textcolor[rgb]{0.64,0.00,0.00}{\textbf{#1}}}
\newcommand{\ExtensionTok}[1]{#1}
\newcommand{\FloatTok}[1]{\textcolor[rgb]{0.00,0.00,0.81}{#1}}
\newcommand{\FunctionTok}[1]{\textcolor[rgb]{0.13,0.29,0.53}{\textbf{#1}}}
\newcommand{\ImportTok}[1]{#1}
\newcommand{\InformationTok}[1]{\textcolor[rgb]{0.56,0.35,0.01}{\textbf{\textit{#1}}}}
\newcommand{\KeywordTok}[1]{\textcolor[rgb]{0.13,0.29,0.53}{\textbf{#1}}}
\newcommand{\NormalTok}[1]{#1}
\newcommand{\OperatorTok}[1]{\textcolor[rgb]{0.81,0.36,0.00}{\textbf{#1}}}
\newcommand{\OtherTok}[1]{\textcolor[rgb]{0.56,0.35,0.01}{#1}}
\newcommand{\PreprocessorTok}[1]{\textcolor[rgb]{0.56,0.35,0.01}{\textit{#1}}}
\newcommand{\RegionMarkerTok}[1]{#1}
\newcommand{\SpecialCharTok}[1]{\textcolor[rgb]{0.81,0.36,0.00}{\textbf{#1}}}
\newcommand{\SpecialStringTok}[1]{\textcolor[rgb]{0.31,0.60,0.02}{#1}}
\newcommand{\StringTok}[1]{\textcolor[rgb]{0.31,0.60,0.02}{#1}}
\newcommand{\VariableTok}[1]{\textcolor[rgb]{0.00,0.00,0.00}{#1}}
\newcommand{\VerbatimStringTok}[1]{\textcolor[rgb]{0.31,0.60,0.02}{#1}}
\newcommand{\WarningTok}[1]{\textcolor[rgb]{0.56,0.35,0.01}{\textbf{\textit{#1}}}}
\usepackage{graphicx}
\makeatletter
\newsavebox\pandoc@box
\newcommand*\pandocbounded[1]{% scales image to fit in text height/width
  \sbox\pandoc@box{#1}%
  \Gscale@div\@tempa{\textheight}{\dimexpr\ht\pandoc@box+\dp\pandoc@box\relax}%
  \Gscale@div\@tempb{\linewidth}{\wd\pandoc@box}%
  \ifdim\@tempb\p@<\@tempa\p@\let\@tempa\@tempb\fi% select the smaller of both
  \ifdim\@tempa\p@<\p@\scalebox{\@tempa}{\usebox\pandoc@box}%
  \else\usebox{\pandoc@box}%
  \fi%
}
% Set default figure placement to htbp
\def\fps@figure{htbp}
\makeatother
\setlength{\emergencystretch}{3em} % prevent overfull lines
\providecommand{\tightlist}{%
  \setlength{\itemsep}{0pt}\setlength{\parskip}{0pt}}
\usepackage{bookmark}
\IfFileExists{xurl.sty}{\usepackage{xurl}}{} % add URL line breaks if available
\urlstyle{same}
\hypersetup{
  pdftitle={BGGN213},
  pdfauthor={Libby Gilmore},
  hidelinks,
  pdfcreator={LaTeX via pandoc}}

\title{BGGN213}
\author{Libby Gilmore}
\date{2025-10-10}

\begin{document}
\maketitle

\subsection{R Markdown - R Basics}\label{r-markdown---r-basics}

\begin{Shaded}
\begin{Highlighting}[]
\CommentTok{\# Tools \textgreater{} Global Options \textgreater{} click "never" saving your workspace makes it less reproducible}

\FunctionTok{library}\NormalTok{(ggplot2)}
\end{Highlighting}
\end{Shaded}

\begin{Shaded}
\begin{Highlighting}[]
\NormalTok{x }\OtherTok{\textless{}{-}}  \DecValTok{1}\SpecialCharTok{:}\DecValTok{5} \CommentTok{\# option and "{-}" generate "\textless{}{-}" shortcut}
\NormalTok{x }\OtherTok{\textless{}{-}}  \DecValTok{1}\SpecialCharTok{:}\DecValTok{50}
\NormalTok{x[}\DecValTok{3}\NormalTok{] }\CommentTok{\# tells you what is the third value in vector x}
\end{Highlighting}
\end{Shaded}

\begin{verbatim}
## [1] 3
\end{verbatim}

\begin{Shaded}
\begin{Highlighting}[]
\NormalTok{letters }\CommentTok{\# preset vector with all lowercase letters in the alphabet}
\end{Highlighting}
\end{Shaded}

\begin{verbatim}
##  [1] "a" "b" "c" "d" "e" "f" "g" "h" "i" "j" "k" "l" "m" "n" "o" "p" "q" "r" "s"
## [20] "t" "u" "v" "w" "x" "y" "z"
\end{verbatim}

\begin{Shaded}
\begin{Highlighting}[]
\NormalTok{letters[}\DecValTok{7}\NormalTok{] }\CommentTok{\# returns [1] g}
\end{Highlighting}
\end{Shaded}

\begin{verbatim}
## [1] "g"
\end{verbatim}

\begin{Shaded}
\begin{Highlighting}[]
\NormalTok{y }\OtherTok{\textless{}{-}} \FunctionTok{c}\NormalTok{(}\DecValTok{10}\NormalTok{, }\DecValTok{100}\NormalTok{, }\DecValTok{4}\NormalTok{)}
\NormalTok{y}
\end{Highlighting}
\end{Shaded}

\begin{verbatim}
## [1]  10 100   4
\end{verbatim}

\begin{Shaded}
\begin{Highlighting}[]
\NormalTok{x }\OtherTok{\textless{}{-}}  \DecValTok{1}\SpecialCharTok{:}\DecValTok{5}
\NormalTok{x }\SpecialCharTok{+} \DecValTok{100}
\end{Highlighting}
\end{Shaded}

\begin{verbatim}
## [1] 101 102 103 104 105
\end{verbatim}

\begin{Shaded}
\begin{Highlighting}[]
\NormalTok{letters[}\FunctionTok{c}\NormalTok{(}\DecValTok{2}\NormalTok{,}\DecValTok{6}\NormalTok{)]}
\end{Highlighting}
\end{Shaded}

\begin{verbatim}
## [1] "b" "f"
\end{verbatim}

\begin{Shaded}
\begin{Highlighting}[]
\NormalTok{x }\SpecialCharTok{+} \FunctionTok{c}\NormalTok{(}\DecValTok{100}\NormalTok{, }\DecValTok{1}\NormalTok{) }\CommentTok{\# adds 100 to an element, then 1 to the next element, and does so recursively. Produces a warning message but not an error message.}
\end{Highlighting}
\end{Shaded}

\begin{verbatim}
## Warning in x + c(100, 1): longer object length is not a multiple of shorter
## object length
\end{verbatim}

\begin{verbatim}
## [1] 101   3 103   5 105
\end{verbatim}

\begin{Shaded}
\begin{Highlighting}[]
\CommentTok{\# numeric {-}{-}{-} doubles and integers}


\CommentTok{\# logicals (boolean) {-}{-}{-} TRUE or FALSE}
\NormalTok{n }\OtherTok{\textless{}{-}} \FunctionTok{c}\NormalTok{(}\ConstantTok{TRUE}\NormalTok{, }\ConstantTok{FALSE}\NormalTok{, T, F)}
\NormalTok{n}
\end{Highlighting}
\end{Shaded}

\begin{verbatim}
## [1]  TRUE FALSE  TRUE FALSE
\end{verbatim}

\begin{Shaded}
\begin{Highlighting}[]
\NormalTok{n }\SpecialCharTok{+} \DecValTok{100} \CommentTok{\# True is interpreted as 1, False is interpreted as 0; so adding 100 adds the value}
\end{Highlighting}
\end{Shaded}

\begin{verbatim}
## [1] 101 100 101 100
\end{verbatim}

\begin{Shaded}
\begin{Highlighting}[]
\FunctionTok{sum}\NormalTok{(n) }\CommentTok{\# sums to 2}
\end{Highlighting}
\end{Shaded}

\begin{verbatim}
## [1] 2
\end{verbatim}

\begin{Shaded}
\begin{Highlighting}[]
\CommentTok{\# this is an example of coercion; logicals can be coerced to numerics}
\NormalTok{x }\SpecialCharTok{\textgreater{}} \DecValTok{3}
\end{Highlighting}
\end{Shaded}

\begin{verbatim}
## [1] FALSE FALSE FALSE  TRUE  TRUE
\end{verbatim}

\begin{Shaded}
\begin{Highlighting}[]
\FunctionTok{sum}\NormalTok{(x}\SpecialCharTok{\textgreater{}}\DecValTok{2}\NormalTok{) }\CommentTok{\# will tell you how many trues you have}
\end{Highlighting}
\end{Shaded}

\begin{verbatim}
## [1] 3
\end{verbatim}

\begin{Shaded}
\begin{Highlighting}[]
\CommentTok{\# characters {-}{-}{-} individual letters or full on strings}
\NormalTok{y }\OtherTok{\textless{}{-}} \FunctionTok{c}\NormalTok{(}\StringTok{"barry"}\NormalTok{,}\StringTok{"monika"}\NormalTok{, }\StringTok{"chandra"}\NormalTok{) }\CommentTok{\# produces an error if chandra has not quotes bc it reads it as a function}
\CommentTok{\#y + 100 \# cannot do math with characters; BUT they are still vectorized so you can use some things; for e.g.:}
\FunctionTok{paste}\NormalTok{(y, }\StringTok{"loves R"}\NormalTok{) }\CommentTok{\# don\textquotesingle{}t have to explicity loop through y, bc it is vectorized}
\end{Highlighting}
\end{Shaded}

\begin{verbatim}
## [1] "barry loves R"   "monika loves R"  "chandra loves R"
\end{verbatim}

\begin{Shaded}
\begin{Highlighting}[]
\FunctionTok{paste}\NormalTok{(y,}\StringTok{"DOES NOT"}\NormalTok{, }\StringTok{"love R"}\NormalTok{)}
\end{Highlighting}
\end{Shaded}

\begin{verbatim}
## [1] "barry DOES NOT love R"   "monika DOES NOT love R" 
## [3] "chandra DOES NOT love R"
\end{verbatim}

\begin{Shaded}
\begin{Highlighting}[]
\NormalTok{z }\OtherTok{\textless{}{-}} \FunctionTok{c}\NormalTok{(}\DecValTok{100}\NormalTok{,}\DecValTok{1}\NormalTok{,}\StringTok{"barry"}\NormalTok{)}
\NormalTok{z}
\end{Highlighting}
\end{Shaded}

\begin{verbatim}
## [1] "100"   "1"     "barry"
\end{verbatim}

\begin{Shaded}
\begin{Highlighting}[]
\CommentTok{\# example of how numerics can be coerced to characters}
\end{Highlighting}
\end{Shaded}

\begin{Shaded}
\begin{Highlighting}[]
\CommentTok{\#data.frame() \# function to make dfs generally will probably not use{-}{-}{-}{-}}
\NormalTok{df }\OtherTok{\textless{}{-}} \FunctionTok{data.frame}\NormalTok{(}\AttributeTok{n=}\DecValTok{1}\SpecialCharTok{:}\DecValTok{5}\NormalTok{, }\AttributeTok{s=}\NormalTok{letters[}\DecValTok{1}\SpecialCharTok{:}\DecValTok{5}\NormalTok{], }\AttributeTok{l=}\FunctionTok{c}\NormalTok{(T,T,F,T,F))}
\NormalTok{df}
\end{Highlighting}
\end{Shaded}

\begin{verbatim}
##   n s     l
## 1 1 a  TRUE
## 2 2 b  TRUE
## 3 3 c FALSE
## 4 4 d  TRUE
## 5 5 e FALSE
\end{verbatim}

\begin{Shaded}
\begin{Highlighting}[]
\NormalTok{df[}\DecValTok{2}\NormalTok{,}\DecValTok{2}\NormalTok{] }\CommentTok{\# [row 2, col 2]}
\end{Highlighting}
\end{Shaded}

\begin{verbatim}
## [1] "b"
\end{verbatim}

\begin{Shaded}
\begin{Highlighting}[]
\NormalTok{df[}\DecValTok{2}\NormalTok{,] }\CommentTok{\# returns entire row 2}
\end{Highlighting}
\end{Shaded}

\begin{verbatim}
##   n s    l
## 2 2 b TRUE
\end{verbatim}

\begin{Shaded}
\begin{Highlighting}[]
\NormalTok{df[,}\DecValTok{3}\NormalTok{] }\CommentTok{\# returns all of column 3}
\end{Highlighting}
\end{Shaded}

\begin{verbatim}
## [1]  TRUE  TRUE FALSE  TRUE FALSE
\end{verbatim}

\begin{Shaded}
\begin{Highlighting}[]
\NormalTok{df[}\FunctionTok{c}\NormalTok{(}\DecValTok{2}\NormalTok{,}\DecValTok{4}\NormalTok{),}\DecValTok{3}\NormalTok{] }\CommentTok{\# print second and fourth row in the third column}
\end{Highlighting}
\end{Shaded}

\begin{verbatim}
## [1] TRUE TRUE
\end{verbatim}

\begin{Shaded}
\begin{Highlighting}[]
\NormalTok{df}\SpecialCharTok{$}\NormalTok{l[}\FunctionTok{c}\NormalTok{(}\DecValTok{2}\NormalTok{,}\DecValTok{4}\NormalTok{)] }\CommentTok{\# gets third column (l), and the second and 4th row}
\end{Highlighting}
\end{Shaded}

\begin{verbatim}
## [1] TRUE TRUE
\end{verbatim}

\begin{Shaded}
\begin{Highlighting}[]
\CommentTok{\# key point is the $ syntax accesses all the columns}

\CommentTok{\# Q. print out values of "s" where n\textgreater{}3 {-}{-}{-}{-}{-}{-}{-}{-}}
\NormalTok{df}\SpecialCharTok{$}\NormalTok{n}
\end{Highlighting}
\end{Shaded}

\begin{verbatim}
## [1] 1 2 3 4 5
\end{verbatim}

\begin{Shaded}
\begin{Highlighting}[]
\NormalTok{df}\SpecialCharTok{$}\NormalTok{n }\SpecialCharTok{\textgreater{}} \DecValTok{3} \CommentTok{\# returns a logical}
\end{Highlighting}
\end{Shaded}

\begin{verbatim}
## [1] FALSE FALSE FALSE  TRUE  TRUE
\end{verbatim}

\begin{Shaded}
\begin{Highlighting}[]
\NormalTok{df[(df}\SpecialCharTok{$}\NormalTok{n }\SpecialCharTok{\textgreater{}} \DecValTok{3}\NormalTok{),}\DecValTok{2}\NormalTok{]}
\end{Highlighting}
\end{Shaded}

\begin{verbatim}
## [1] "d" "e"
\end{verbatim}

\begin{Shaded}
\begin{Highlighting}[]
\CommentTok{\# Alternatives}
\NormalTok{inds }\OtherTok{\textless{}{-}}\NormalTok{ df}\SpecialCharTok{$}\NormalTok{n }\SpecialCharTok{\textgreater{}} \DecValTok{3}
\NormalTok{inds}
\end{Highlighting}
\end{Shaded}

\begin{verbatim}
## [1] FALSE FALSE FALSE  TRUE  TRUE
\end{verbatim}

\begin{Shaded}
\begin{Highlighting}[]
\NormalTok{df[inds,]}
\end{Highlighting}
\end{Shaded}

\begin{verbatim}
##   n s     l
## 4 4 d  TRUE
## 5 5 e FALSE
\end{verbatim}

\begin{Shaded}
\begin{Highlighting}[]
\NormalTok{df[inds,]}\SpecialCharTok{$}\NormalTok{s}
\end{Highlighting}
\end{Shaded}

\begin{verbatim}
## [1] "d" "e"
\end{verbatim}

\begin{Shaded}
\begin{Highlighting}[]
\NormalTok{df[inds,}\DecValTok{2}\NormalTok{]}
\end{Highlighting}
\end{Shaded}

\begin{verbatim}
## [1] "d" "e"
\end{verbatim}

\begin{Shaded}
\begin{Highlighting}[]
\NormalTok{df[(df}\SpecialCharTok{$}\NormalTok{n }\SpecialCharTok{\textgreater{}} \DecValTok{3}\NormalTok{),}\DecValTok{2}\NormalTok{]}
\end{Highlighting}
\end{Shaded}

\begin{verbatim}
## [1] "d" "e"
\end{verbatim}

\begin{Shaded}
\begin{Highlighting}[]
\FunctionTok{source}\NormalTok{(}\StringTok{"http://thegrantlab.org/misc/cdc.R"}\NormalTok{)}

\CommentTok{\#q1.}
\FunctionTok{tail}\NormalTok{(cdc, }\DecValTok{20}\NormalTok{)}
\end{Highlighting}
\end{Shaded}

\begin{verbatim}
##         genhlth exerany hlthplan smoke100 height weight wtdesire age gender
## 19981 excellent       1        1        1     75    195      195  36      m
## 19982 very good       0        1        1     74    210      210  40      m
## 19983 very good       1        0        1     63    171      130  31      f
## 19984 very good       1        1        1     71    190      180  41      m
## 19985      good       1        1        1     69    180      160  57      m
## 19986 very good       1        1        1     64    120      115  71      f
## 19987 excellent       1        1        1     63    140      115  38      f
## 19988      good       1        1        0     72    200      195  40      m
## 19989 very good       1        1        1     74    230      190  53      m
## 19990      good       1        1        0     73    230      200  38      m
## 19991 excellent       1        1        0     71    195      190  43      m
## 19992 very good       1        1        1     72    210      175  52      m
## 19993 very good       1        1        0     71    180      180  36      m
## 19994 very good       0        1        1     63    165      120  31      f
## 19995      good       0        1        1     69    224      224  73      m
## 19996      good       1        1        0     66    215      140  23      f
## 19997 excellent       0        1        0     73    200      185  35      m
## 19998      poor       0        1        0     65    216      150  57      f
## 19999      good       1        1        0     67    165      165  81      f
## 20000      good       1        1        1     69    170      165  83      m
\end{verbatim}

\begin{Shaded}
\begin{Highlighting}[]
\CommentTok{\#q2}
\FunctionTok{ggplot}\NormalTok{(}\AttributeTok{data=}\NormalTok{cdc, }\FunctionTok{aes}\NormalTok{(}\AttributeTok{x=}\NormalTok{height, }\AttributeTok{y=}\NormalTok{weight)) }\SpecialCharTok{+}
  \FunctionTok{geom\_point}\NormalTok{() }\SpecialCharTok{+}
  \FunctionTok{xlab}\NormalTok{(}\StringTok{"Height (inches)"}\NormalTok{) }\SpecialCharTok{+} 
  \FunctionTok{ylab}\NormalTok{(}\StringTok{"Weight (pounds)"}\NormalTok{) }\SpecialCharTok{+}
  \FunctionTok{theme\_bw}\NormalTok{()}
\end{Highlighting}
\end{Shaded}

\pandocbounded{\includegraphics[keepaspectratio]{BGGN213_course4_files/figure-latex/unnamed-chunk-2-1.pdf}}

\begin{Shaded}
\begin{Highlighting}[]
\CommentTok{\#q4}
\FunctionTok{cor}\NormalTok{(cdc}\SpecialCharTok{$}\NormalTok{height, cdc}\SpecialCharTok{$}\NormalTok{weight)}
\end{Highlighting}
\end{Shaded}

\begin{verbatim}
## [1] 0.5553222
\end{verbatim}

\begin{Shaded}
\begin{Highlighting}[]
\NormalTok{x }\OtherTok{\textless{}{-}} \DecValTok{1}\SpecialCharTok{:}\DecValTok{50}

\FunctionTok{plot}\NormalTok{(x, }\AttributeTok{col=}\StringTok{"purple"}\NormalTok{)}
\end{Highlighting}
\end{Shaded}

\pandocbounded{\includegraphics[keepaspectratio]{BGGN213_course4_files/figure-latex/unnamed-chunk-3-1.pdf}}

\begin{Shaded}
\begin{Highlighting}[]
\FunctionTok{plot}\NormalTok{(}\FunctionTok{cos}\NormalTok{(x), }\FunctionTok{sin}\NormalTok{(x), }\AttributeTok{typ=}\StringTok{"l"}\NormalTok{, }\AttributeTok{col=}\StringTok{"red"}\NormalTok{, }\AttributeTok{lwd=}\DecValTok{3}\NormalTok{)}
\end{Highlighting}
\end{Shaded}

\pandocbounded{\includegraphics[keepaspectratio]{BGGN213_course4_files/figure-latex/unnamed-chunk-3-2.pdf}}

\begin{Shaded}
\begin{Highlighting}[]
\FunctionTok{log}\NormalTok{(}\DecValTok{10}\NormalTok{)}
\end{Highlighting}
\end{Shaded}

\begin{verbatim}
## [1] 2.302585
\end{verbatim}

\begin{Shaded}
\begin{Highlighting}[]
\FunctionTok{log}\NormalTok{(}\DecValTok{10}\NormalTok{,}\AttributeTok{base=}\DecValTok{10}\NormalTok{)}
\end{Highlighting}
\end{Shaded}

\begin{verbatim}
## [1] 1
\end{verbatim}

\end{document}
